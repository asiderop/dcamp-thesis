\chapter{Future Work}
\label{future_work}

\begin{itemize}

\item multiple-level branches are not supported in the current implementation. that is, all collector nodes have the
      root node as their parent and only have leaf nodes as their children.
      \begin{itemize}
      \item support would allow large group configurations to be automatically split into multiple (identically
            configured) branches for improved scalability
      \end{itemize}

\item network failure: \dcamp does not support any fault tolerance for network failures; \dcamp only attempts to recover
from node failures. It is assumed that if (part of) the network goes down, the lack of data from that subnet will
suffice. specifically, \dcamp does not tolerate the split brain syndrome\cite{needed}.

\item time accuracy: The system time among multiple nodes in the system may vary significantly; \dcamp is not meant to
be a high-resolution system with respect to the order of performance data occurrences. It is assumed that NTP provides
sufficient time synchronization across all nodes in the system OR the precise ordering of performance events in the
system is not required.

\item use IOLoop (green events?) instead of polling sockets for messages
      \begin{itemize}
      \item look into using an io loop with callbacks to better manage sockets
      \item possibly, use a single io loop, hosted by the base node service and passed to the roles it starts?
      \item this could greatly increase the scalability of dCAMP and reduce its overhead
      \end{itemize}


\end{itemize}

