\chapter{Implementation}
\label{implementation}

\section{Requirements}

\subsection{Functional}

\begin{enumerate}

\item Configuration
      \begin{itemize}
      \item instantiation, administration
      \item topology coordination
      \item metric collection spec.
      \end{itemize}

\item Metric Collection
      \begin{itemize}
      \item \dcamp API on top of CAMP
      \item filters (at various levels), thresholds
      \item aggregation of metrics across nodes
      \item output to log file
      \end{itemize}

\item Fault Tolerance
      \begin{itemize}
      \item simple rules to handle failures
      \end{itemize}

\end{enumerate}

\subsection{Fault Tolerance}

\begin{enumerate}

\item Topology MUST sustain brief network disconnectivity of any node.
\item Sensor nodes MUST be allowed to enter/exit topology at any time.
\item Root/Collector nodes (i.e. parents) MUST failover in case of extended disconnectivity.
      \begin{itemize}
      \item Loss of previously collected data SHOULD be minimized during failover.
      \end{itemize}
\item Management node SHOULD be allowed to enter/exit topology at any time.

\end{enumerate}

\subsection{Testing}

\begin{itemize}

\item \textbf{Transparency:} \dcamp SHOULD introduce negligible performance impact on sensor nodes.
\item \textbf{Accuracy:} \dcamp MUST accurately report metrics/performance of sensor nodes (individual and aggregated).
\item \textbf{Scalability:} \dcamp SHOULD maintain its transparency and accuracy as it scales (i.e. the number of sensor
      nodes increases).
\item \textbf{Fault Tolerance:} \dcamp MUST successfully handle entrance/exit of any node(s) in the system.

\end{itemize}

\section{\dcamp Operation}

\subsection{Sequence of \dcamp Operation}
\label{operation_sequnce}

The following steps describe how the \dcamp system is turned on. The Base nodes (other than the node assigned to be the
Root) can be started at any time by using the \dcamp CLI, before or after the Root node is initialized. It is expected
these Base nodes are managed by a watchdog utility which automatically restarts the node if it exits for any reason.

\begin{figure}[H]
\vspace{+10pt}
\begin{lstlisting}[language=bash,frame=single,basicstyle=\footnotesize\ttfamily]
#!/usr/bin/env bash
while [ true ]
do
    dcamp base --address localhost:56789
done
\end{lstlisting}
\vspace{-10pt}
\caption{Sample Watchdog Script}
\label{fig:sample_watchdog}
\end{figure}

\begin{enumerate}

\item User promotes a Root node via the \dcamp CLI, specifying a configuration file and a Base node's address.
\item Root node connects to each Base node and begins the ``discover'' \hyperref[proto_topo]{Topology Protocol}.
\item Base nodes join the \dcamp system at any time, being assigned as Collector or Metric nodes in the topology.

\item \dcamp runs in a steady state, nodes entering or exiting the system at any time.

      \begin{itemize}
      \item Performance counters are sampled, filtered, reported, and logged by the Metric nodes at regular intervals
            according to the \hyperref[configuration]{\dcamp Configuration}.
      \item Performance counters received from child nodes are aggregated, filtered, reported, and logged by Collector
            nodes at regular intervals according to the \dcamp Configuration.
      \item Performance counters received from child nodes are aggregated and logged by Root node for later processing
            (e.g. graphing metrics during a test scenario or correlating statistics with a distributed event log).
      \end{itemize}

\item User stops \dcamp by using the \dcamp CLI command.
\item Root node begins the ``stop'' Topology Protocol.
\item Collector and Metric nodes exit the topology and revert to Base nodes.
\item Root node exits, reverting to Base node.

\end{enumerate}

\subsection{Threading Model}

As mentioned above as the first and third steps of \dcamp operation, a Base node can transform into one of the three
active \dcamp roles: Root, Collector, or Metric. This transformation is actually the Base role (via the Node service)
launching and managing another Role internally. This interaction is depicted in Figure
\ref{fig:node_role_service_image}.

\begin{figure}[H]
    \centering
    \includegraphics[scale=0.5]{node-role-service.pdf}
    \caption[Node, Role, Services Threading Model Diagram]
            {Node, Role, Services Threading Model Diagram: Thread boundaries are represented by dashed lines. Except for
	     the Node service's \texttt{SUB} and \texttt{REQ} sockets, all arrows represent \texttt{PAIR} socket
	     communication.}
    \label{fig:node_role_service_image}
\end{figure}

When a Base node is running, only the bottom two threads (the Base role and the Node service) are active. Once it
receives an assignment from the ``discover'' Topology Protocol or the \dcamp CLI, the Node service launches an
appropriate Role thread which, in turn, launches one or more Role-specific service threads.

All communication between the roles and services occurs across \texttt{PAIR} control sockets. There are also various
service-to-service communications which occur via \texttt{inproc} transport sockets (e.g. the internal
\hyperref[proto_data]{Data Flow Protocol}) and shared memory data structures (e.g. the Configuration service).

Also mentioned in section \ref{operation_sequnce} as the last two steps, each Role exits and, by doing so, reverts
itself back to a Base node. This is handled just like before, with the Node service receiving a \texttt{STOP} message
via the ``stop'' Topology Protocol and then notifying the internally running Role to shut down. The Role thread then
notifies its service threads, waits for them to finish, then exits.

\section{ZeroMQ Protocols}

For a quick background on ZeroMQ socket types and message patterns, please see Appendix \ref{zeromq_primer}.

\subsection{Topology Protocol}
\label{proto_topo}

The \dcamp distributed topology is dynamically established as the Root node sends out its discovery message and receives
join messages from Base nodes. Once a Base node responds to the Root, the Base node is given its assignment.
Additionally, the Root node can shutdown the system using this same protocol but responding with a \texttt{STOP} message
instead of an assignment.

\begin{figure}[H]
\vspace{+10pt}
\begin{verbatim}
topo-discovery = *discover join
discover       = R-MARCO
join           = B-POLO ( R-ASSIGN / R-STOP / R-WTF )
\end{verbatim}
\vspace{-5pt}
\caption[Topology Protocol]
        {Topology Protocol: \texttt{R-} represents the Root node sending a message and \texttt{B-}
         represents a Base node sending a message.}
\label{fig:proto_topo_spec}
\end{figure}

\begin{figure}[H]
    \centering
    \includegraphics[scale=0.5]{topo.pdf}
    \label{fig:proto_topo_image}
    \caption{Topology Protocol Diagram}
\end{figure}

\subsubsection{Message Definitions}

\textbf{\texttt{TOPO}} is a generic topology message consisting of four frames. This message type is designed to be sent
across a PUB/SUB connection, from which subscribers filter incoming messages using the first frame. This design proves
useful for the \hyperref[proto_reco]{Recovery Protocols}.

The \texttt{MARCO} message is simply shorthand for \texttt{TOPO(key="/MARCO")}.

\begin{figure}[H]
\vspace{+10pt}
\begin{verbatim}
Frame 0: key, as 0MQ string
Frame 1: root address, as 0MQ string
Frame 2: root UUID, 16 bytes in network order
Frame 3: <empty> or content, as 0MQ string
\end{verbatim}
\vspace{-20pt}
\caption{\texttt{TOPO} Message Definition}
\label{fig:message_topo}
\end{figure}

\textbf{\texttt{CONTROL}} is a generic control message consisting of four frames and designed to be sent across a
REQ/REP connection. The \texttt{POLO}, \texttt{ASSIGN}, and \texttt{STOP} messages are shorthand for
\texttt{CONTROL(command="POLO")}, \texttt{CONTROL(command="ASSIGN")}, and \texttt{CONTROL(command="STOP")} respectively.

In the case of \texttt{ASSIGN}, the third frame contains the specific topology instructions (level-one collector, leaf
node, etc.) being sent to the Base node.

\begin{figure}[H]
\vspace{+10pt}
\begin{verbatim}
Frame 0: command, as 0MQ string
Frame 1: base address, as 0MQ string
Frame 2: base UUID, 16 bytes in network order
Frame 3: properties, JSON-encoded, as 0MQ string

command     = "polo" / "assignment"
properties  = *( parent / level / group )
parent      = "parent=" <node-address>
level       = "level=" ( "root" / "branch" / "leaf" )
group       = "group=" <group-identity>
\end{verbatim}
\vspace{-20pt}
\caption{\texttt{CONTROL} Message Definition}
\label{fig:message_control}
\end{figure}

\textbf{\texttt{WTF}} is \dcamp's error message type. It has three frames (though Frame 2 may be empty) with the first
designed to make error detection simple.

\begin{figure}[H]
\vspace{+10pt}
\begin{verbatim}
Frame 0: "WTF", as 0MQ string
Frame 1: error code, 4 bytes in network order
Frame 2: <empty> or error message, as 0MQ string
\end{verbatim}
\vspace{-20pt}
\caption{\texttt{WTF} Message Definition}
\label{fig:message_wtf}
\end{figure}

\subsection{Configuration Replication Protocol}
\label{proto_config}

The \dcamp configuration distribution algorithm adheres to the Clustered Hashmap Protocol\cite{chp} with a few minor
(and one major) modifications:

\begin{figure}[ht]
    \centering
    \includegraphics[scale=0.66]{config.png}
    \caption{Configuration Protocol Diagram}
    \label{fig:proto_config_image}
\end{figure}

\begin{enumerate}
\item only the Root node may write new configuration changes,
\item the full configuration table will be replicated across all first-level Collector nodes (lower-level nodes may
      filter their configuration to only store relevant data),
\item a different set of command names are used (as described below), and
\item configuration updates are distributed via the \dcamp hierarchy (instead of directly from the Root node).
\end{enumerate}

A newly assigned first-level Collector node will first subscribe to new configuration updates from the Root node and
then send a configuration snapshot request to the Root node. A newly assigned Sensor (or non-first-level Collector) node
will first subscribe to new configuration updates from its parent Collector node, and then send its parent Collector
node a filtered configuration snapshot request. Once its snapshot has been successfully received, a node will process
any pending configuration updates and then, in the case of a Collector node, respond to child node snapshot requests.

\begin{figure}[ht]
\vspace{+10pt}
\begin{verbatim}
config-distribution = *update / snap-sync
update              = P-KVPUB / P-HUGZ
snap-sync           = C-ICANHAZ *P-KVSYNC ( P-KTHXBAI / P-WTF )
\end{verbatim}
\vspace{-5pt}
\caption[Configuration Protocol Specification]
	{Configuration Protocol Specification: \texttt{P-} represents the parent node (Root or Collector) sending a
	 message and \texttt{C-} represents the child node sending a message.}
\label{fig:proto_config_spec}
\end{figure}

\subsubsection{Message Definitions}

These messages come from the CHP protocol. Additionally, a WTF error message may be sent by the parent in case of error.
It should be noted, each of the following messages is really the same five-frame format with varying key values and
semantics.

\textbf{ICANHAZ} \\
configuration snapshot request

\begin{verbatim}
Frame 0: "ICANHAZ"
Frame 1: sequence number, 8 bytes in network order
Frame 2: <empty>
Frame 3: <empty>
Frame 4: subtree specification
\end{verbatim}

\textbf{KVSYNC} \\
snapshot sync message

\begin{verbatim}
Frame 0: key, as 0MQ string
Frame 1: sequence number, 8 bytes in network order
Frame 2: <empty>
Frame 3: <empty>
Frame 4: value, as blob
\end{verbatim}

\textbf{KTHXBAI} \\
end of successful snapshot sync

\begin{verbatim}
Frame 0: "KTHXBAI"
Frame 1: sequence number, 8 bytes in network order
Frame 2: <empty>
Frame 3: <empty>
Frame 4: subtree specification
\end{verbatim}

\textbf{KVPUB} \\
configuration update sent from parent to child

\begin{verbatim}
Frame 0: key, as 0MQ string
Frame 1: sequence number, 8 bytes in network order
Frame 2: UUID, 16 bytes in network order
Frame 3: properties, JSON-encoded, as 0MQ string
Frame 4: value, as blob

properties = ...
\end{verbatim}

\textbf{HUGZ} \\
heartbeat message sent from parent to child (when no config updates)

\begin{verbatim}
Frame 0: "HUGZ"
Frame 1: 00000000
Frame 2: <empty>
Frame 3: <empty>
Frame 4: <empty>
\end{verbatim}

\subsection{Data Flow Protocol}
\label{proto_data}

There are two data flow protocols in the \dcamp system: the external protocol for data flowing from one node to the next
(via PUB/SUB) and the internal protocol for data flowing between components of a single node (via PUSH/PULL). Both
protocols have the same specification and use the same message formats.

\begin{figure}[H]
    \centering
    \includegraphics[scale=0.5]{data.pdf}
    \caption{Data Flow Diagram}
    \label{fig:proto_data_image}
\end{figure}

The \dcamp data flow protocol is very simple, comprised of a single data message type. The data flows from one node to
another via PUB/SUB sockets. Internally, data flows from the upstream data producers, through a filtering/processing
unit, and out to downstream data consumers via PUSH/PULL sockets.

When data rate is slower than a predefined threshold, heartbeats are sent instead to keep inter-node connections alive.

\begin{figure}[H]
\vspace{+10pt}
\begin{verbatim}
data-flow = *( METRIC / HUGZ )
\end{verbatim}
\vspace{-5pt}
\caption[Data Flow Specification]
	{Data Flow Specification: All messages are sent from child (Metric or Collector) to parent (Collector or Root).}
\label{fig:proto_data_spec}
\end{figure}

\subsubsection{Performance Measurement}

When discussing performance measurement, it is important to understand how metrics are sampled, calculated, and
presented to an end user.

Performance metrics, also called counters, are usually monotonically increasing values. That is, reading its raw,
instantaneous value is virtually meaningless; to correctly read the counter it must be sampled at two different points
in time and then calculated.

For example, when displaying a graph of data point for non-basic metric types, each data point is really a calculated
value of the value at the current timestamp and that at the previous timestamp. It is possible to look at fewer data
samples to first get a course-grain view (e.g. five-minute samples) of the metric before drilling in a looking at
finer-grain samples (e.g. one-second samples).

Non-monotonically increasing counters do exist (e.g. disk speed, Ethernet uplink speed, etc.), but these are usually
fairly static configuration values and do not need to be sampled frequently. \dcamp supports these types of counters
with the "basic" metric type.

Table \ref{tab:metric_types} shows how each of the \dcamp metric types are calculated. Note: unlike some other
performance measurement frameworks\cite{ganglia}, \dcamp stores all metrics in their raw, uncalculated form and only
presents a calculated value upon display.

\begin{table}
\begin{tabular}{|l|l|l|}
\hline
\textbf{Type} & \textbf{Contents of Single Sample} & \textbf{Calculation of Two Samples}
\\
\hline
basic & raw value at specified timestamp & \( C = V_{t_2} \)
\\
\hline
delta & raw value at specified timestamp & \( C = V_{t_2} - V_{t_1} \)
\\
\hline
rate & raw value at timestamp & \( C = (V_{t_2} - V_{t_1}) / (t_2 - t_1) \)
\\
\hline
average & raw value and raw base value at timestamp & \( C = (V_{t_2} - V_{t_1}) / (B_{t_2} - B_{t_1}) \)
\\
\hline
percent & raw value and raw base value at timestamp & \( C = ( (V_{t_2} - V_{t_1}) / (B_{t_2} - B_{t_1}) ) 100 \)
\\
\hline
\end{tabular}
\caption[Metric Types]
        {Metric Types: \(C\) represents the value calculated from two samples taken at \(t_1\) and \(t_2\). \(V\) is the
	 value and \(B\) is the base value in the \texttt{METRIC} message}
\label{tab:metric_types}
\end{table}

\subsubsection{Message Definitions}

\textbf{\texttt{METRIC}} is a five-frame message containing the performance metric data sampled by the Sensor service or
calculated by the Aggregation service. The \texttt{HUGZ} message is simply shorthand for \texttt{METRIC(type="HUGZ")}.

A single data sample MUST contain: source identifier (node or aggregation), metric identifier, timestamp, and one or two
values depending on the metric type.

In case of \texttt{HUGZ}, no other property strings are used, and Frames 3 through 5 are all empty. Frame 4 will be
non-empty for average and percent types.

NOTE: This message needs to be cleaned up...its a bit too verbose. I think just the config-seqid is needed to identify
      the metric being sampled.

\begin{figure}[H]
\vspace{+10pt}
\begin{verbatim}
Frame 0: data source (leaf or collector node endpoint), as 0MQ string
Frame 1: properties, JSON-encoded as 0MQ string
Frame 2: time in ms epoch utc, 8 bytes in network order
Frame 3: value, 8 bytes in network order
Frame 4: base value, 8 bytes in network order; only for average and percent types

properties = *( type / detail / config / seqid )
type       = "type=" ( "HUGZ" / "basic" / "delta" / "rate" / "average" / "percent" )
detail     = "detail=" <string>
config     = "config-name=" <string>
seqid      = "config-seqid=" <integer>
\end{verbatim}
\vspace{-20pt}
\caption{\texttt{METRIC} Message Definition}
\label{fig:message_metric}
\end{figure}

\subsection{Recovery Protocols}
\label{proto_reco}

The \dcamp Recovery Protocols are used for the \hyperref[algor_promo]{Promotion} and \hyperref[algor_elect]{Election}
algorithms and use the same base messages as the \hyperref[proto_topo]{Topology Protocol}, \texttt{TOPO} and
\texttt{CONTROL}.

\begin{figure}[H]
\vspace{+10pt}
\begin{verbatim}
branch-recovery  = *sos group-stop
sos              = B-SOS
group-stop       = R-GROUP B-POLO R-STOP
\end{verbatim}
\vspace{-5pt}
\caption[Branch Recovery Protocol]
	{Branch Recovery Protocol: \texttt{R-} represents the Root node sending a message and \texttt{B-} represents a
	 Base node sending a message.}
\label{fig:proto_reco_branch_spec}
\end{figure}

The Branch Recovery Protocol is initiated by Base nodes when they detect their Collector has died. Once the Root node
has received an \texttt{SOS} message from at least one third of the branch's Base nodes, the Root proceeds to shutdown
the entire branch using the Topology Protocol.

The \texttt{GROUP} message is shorthand for the Topology Protocol's \texttt{TOPO} message with a key value of
\texttt{"/GROUP/<group-name>"}. This takes advantage of ZeroMQ's Pub-Sub filtering to only stop the faulty branch.

\begin{figure}[H]
\vspace{+10pt}
\begin{verbatim}
root-recovery  = *election
election       = C-WUTUP *C-YO C-IWIN
\end{verbatim}
\vspace{-5pt}
\caption[Root Recovery Protocol]
        {Root Recovery Protocol: \texttt{C-} represents a Collector node sending a message.}
\label{fig:proto_reco_root_spec}
\end{figure}

As each Collector nodes detect the Root node has died, it attempts to start an election via the Root Recovery Protocol.
The \texttt{WUTUP} and \texttt{IWIN} messages are shorthand for \texttt{TOPO(key="/RECOVERY/wutup"} and
\texttt{TOPO(key="/RECOVERY/iwin"} respectively. The \texttt{YO} message is shorthand for
\texttt{CONTROL(command="yo")}.


