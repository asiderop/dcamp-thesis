\chapter{\dcamp}
\label{dcamp}

The Distributed Common API for Measuring Performance (\dcamp) is a distributed performance framework built on top of
Mark Gabel and Michael Haungs' 2007 research on \emph{CAMP: a common API for measuring performance} \cite{gabel2007}.
The fundamental functionality of \camp is providing an accurate and ``consistent method for retrieving system
performance data from multiple platforms.'' \dcamp\ takes advantage of this functionality and the authors' work done in
validating \camp's accuracy and adds the following core feature sets:

\begin{itemize}
\item Stateful Performance API
\item Distributed Performance Data Aggregation
\item Performance Filters and Triggers
\item Simplistic Fault Tolerance
\end{itemize}

\section{Terminology}
Knowing the following terminology will make it easier to understand and discuss the \dcamp project, its main goals, its
usage, its components, and how it works. 

\begin{description}

\item[Distributed Performance Testing Framework (DPTF),]
\item[Distributed Performance Monitoring Framework (DPMF),]
\item[Distributed Performance Framework (DPF):]

An DPTF or DPMF (collectively termed DPF) is a framework which allows its users to evaluate the performance of a system
from both black box and white box perspectives by publishing interfaces for instrumenting, collecting, analyzing, and
visualizing performance data across the distributed system and distributed applications. Typically, the framework
provides a black box interface into monitoring a distributed system or application and includes mechanisms for
triggering actions based on performance events. The \dcamp project is designed to be an DPF. 

\item[Performance Metric:]
Performance metrics are any data about a given node relating to its throughput, capacity, utilization, or latency. In
\dcamp, these are grouped into four different sets of performance metrics (global, network, disk, and per-process) and a
fifth set of inquiry metrics. They are described fully below. 

\item[Metric Collection,]
\item[Metric Aggregation,]
\item[Metric Calculation:]
Metric collection (or aggregation or calculation) is the process of combining metrics from multiple nodes into a single
metric. Performance metrics, while useful at an individual system granularity, can be rather limited in value for a DPF
where the goal is measurement of the distributed system as a whole. Metric collection provides a coarser granularity
for the performance metrics. Collection could calculate a sum, average, percent, or any other mathematically relevant
operation across multiple nodes in the system. 

\item[Filter,]
\item[Throttle,]
\item[Threshold:]
Filtering (or throttling or thresholding) provides a mechanism for reducing the amount of data sent between nodes of the
system. Filtering allows a user to specify when or at what point to report metrics from one level to its parent. For
example, a filter might be set to only report average CPU utilization that is over seventy-five percent. 

\item[\dcamp Node,]
\item[\dcamp Process:]
A single, independently running instance of \dcamp in the distributed system is called a \dcamp node or process. More
than one node may exist on a single computer. A node consists of the Node role and zero or more other \dcamp roles. 

\item[\dcamp Role:]
Roles are a way of logically grouping functions within the \newline \dcamp system, from performance metric gathering to
\dcamp system management. A description of all the \dcamp roles can be found below. 

\item[\dcamp Service:]
Services in the \dcamp system are implementations of one or more \dcamp roles. There does NOT exist a one-to-one
relationship between services and roles; the \dcamp service-to-role mapping can be seen below. Each services is
implemented in \dcamp as a single Python class. 

\item[\dcamp Hierarchy:]
The \dcamp system is organized in a hierarchical pattern with respect to data movement and system control functionality.
The hierarchy can be thought of as a tree structure, with leaf nodes being at the top of the hierarchy and a single root
node at the bottom. Metric data moves down the hierarchy from leaves to the root; configuration data and control
commands move up from the root to the leaves. 

\item[\dcamp Level:]
Levels are a way of organizing the \dcamp hierarchy horizontally. Levels are defined by their distance from the root
node. For example, level one is one node away from the root node, or said another way, the first level is directly
"connected" to the root node. The second level is two nodes away from the root node, or any node in the second level is
connected to the root node by another node (in the first level). This necessarily means the root is in level zero (all
by itself). 

\item[Parent Node:]
Nodes are called parent nodes if there is at least one node connected to it from a level of higher ordinal value. For
example, a node in level one with at least one node connected to it from level two is considered a parent node. The
root node is inherently a parent node. 

\item[Child Node:]
A node is called a child node if it is connected to another node in a level of lower ordinal value. For example, a node
in level one is connected to the root node (in level zero), so it is called a child node. The root node is the only
node in the \dcamp system which is NOT a child node. 

\item[Configuration:]
The \dcamp configuration specifies everything about the system, including root node, hierarchy levels, metrics, sampling
frequency, reporting frequency, filtering, communication details, etc. The configuration is set at the root node and
then distributed to the rest of the \dcamp system. Configuration details can be found below. 

\item[Metric Sampling:]
Metric sampling is the process of measuring metrics on a given node. 

\item[Metric Reporting:]
Metric reporting is the process of sending sampled metrics to a parent node.

\end{description}

\section{\dcamp Metrics}
Metrics marked with "(\dcamp)" are extensions added by the \dcamp project to the basic CAMP metrics. These provide a
performance view of multiple nodes in the distributed network and are collected by the Aggregate service rather than the
CAMP service.

\subsection{Global Metrics}
Global metrics measure overall CPU, process, thread, and memory usage of the system.
\begin{itemize}
\item Node CPU usage 
\item Node free physical memory (KB) 
\item Aggregate average CPU usage (\dcamp) 
\item Aggregate free physical memory (percentage) (\dcamp)
\end{itemize}

\subsection{Network I/O Metrics}
Network metrics measure utilization of a given network interface on the system.
\begin{itemize}
\item Total bytes sent on the given interface 
\item Total packets sent on the given interface 
\item Total bytes received on the given interface 
\item Total packets received on the given interface 
\item Aggregate bytes sent (\dcamp) 
\item Aggregate packets sent (\dcamp) 
\item Aggregate bytes received (\dcamp) 
\item Aggregate packets received (\dcamp)
\end{itemize}

\subsection{Disk I/O Metrics}
Disk I/O metrics measure throughput of a given disk or partition on the system.
\begin{itemize}
\item Number of read operations on the given disk 
\item Number of write operations on the given disk 
\item Number of read operations on the given partition 
\item Number of write operations on the given partition 
\item Aggregate number of read operations (\dcamp) 
\item Aggregate number of write operations (\dcamp)
\end{itemize}

\subsection{Per-process Metrics}
Per-process metrics measure CPU, memory, and thread usage for a single process on the system.
\begin{itemize}
\item Number of major and minor page faults 
\item Process CPU utilization 
\item Process user mode CPU utilization 
\item Process privileged mode CPU utilization 
\item Size of the process' working set in KB 
\item Size of the used virtual address space in KB 
\item Number of threads contained in the process
\end{itemize}

\subsection{Inquiry Metrics}
Inquiry metrics provide a mechanism for enumerating various properties of the system.
\begin{itemize}
\item Enumeration of the available disk partitions 
\item Enumeration of the available physical disks 
\item Enumeration of the valid inputs to the network functions 
\item Number of CPUs in the system 
\item "process identifier" for the given pid 
\item "process identifier" for each running process launched from an executable of the given name
\end{itemize}

\section{Execution}

These steps are for the current, prototype version of \dcamp. \textit{Setup: All nodes are running as base nodes
(listening for commands via CLI and network port)}

\begin{enumerate}

\item User initializes root node via CLI (providing configuration) 
\item Root node registers each listening node (configuring as metric nodes) 
\item Metrics are sampled, reported, and logged (repeating indefinitely) 
\item Metric nodes sample metrics and report metrics back to root node 
\item Root node logs received metrics to disk 
\item User shuts down root node via CLI 
\item Root node unregisters each listening node (reverting to base nodes) 
\item Root node terminates (reverting to base node)

\end{enumerate}

