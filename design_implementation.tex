\chapter{Design \& Implementation}
\label{design_implementation}

\textbf{General design:} semi-centralized, hierarchical peer-to-peer system utilizing the pipe-flow architecture pattern
in which leaf (sensor) nodes of the hierarchy collect data, filter out extraneous data, and send it up the pipe to an
aggregate node which subsequently filters out more data and sends it up to another aggregate or root node.

\section{\dcamp Roles and Services}

The dCAMP distributed system is comprised of one or more nodes each running one or more roles. Each role--a published,
remotely accessible interface--provides one or more sets of functionality. Each set of functionality is known as a
service.

\subsection{Services}

\begin{itemize}

\item \textbf{Node}---rudimentary dCAMP functionality; handles topology communication, heartbeat monitoring, and failure
recovery.

\item \textbf{Sensor}---local performance metric gathering; essentially the dCAMP layer on top of the OS and hardware
performance APIs (accessed via CAMP).

\item \textbf{Filter}---performance metric filtering; provides throttling and thresholding of metrics.

\item \textbf{Aggregation}--—performance metric aggregation; provides collection of and calculation on metrics from
multiple sensors and/or collectors.

\item \textbf{Management}--—primary entry-point for end-user control of dCAMP distributed system; this is the dCAMP
instrument panel, providing basic administration functions (e.g. start, stop, etc.).

\item \textbf{Configuration}--—complete configuration replication; provides topology and configuration distribution.

\end{itemize}

\subsection{Roles}

The \textit{Base} role must be running on each node for it to be part of the dCAMP distributed system. In this document,
a "Base node" is defined as a dCAMP node which has not yet been configured, i.e. it has not joined a running dCAMP
system.

The \textit{Metric} role runs on the nodes from which performance metrics should be collected. The \textit{Collector}
role acts as an aggregation point in the system, combining performance data from multiple \textit{Metric} (and
\textit{Collector}) nodes and providing additional aggregated performance metrics.

There is only one \textit{Root} role active in the system; it acts as the master copy of the dCAMP configuration and
sole user-interface point. The \textit{Root} role is not strictly attached to any given node in the system. Rather, the
\textit{Root} role may dynamically move to any first-level \textit{Collector} node if the current \textit{Root} node
fails.

Depending on the use case and desired system performance, an administrator may choose to split roles across multiple
nodes or collapse them onto a single node. For example, a single node may act as \textit{Metric}, \textit{Collector},
and \textit{Root} for smaller systems while larger systems would employ dedicated \textit{Collector} nodes.

\subsection{Role-to-Service Mapping}

The following table lists the roles which can be "published" by a dCAMP node and the services which they implement.

\begin{tabular}{|l|l|}

\hline
\textbf{Role} & \textbf{Service(s)} \\
\hline
Root & Management, Aggregation, Filter, Configuration \\
\hline
Collector & Aggregation, Filter, Configuration \\
\hline
Metric & Sensor, Filter, Configuration \\
\hline
Base & Node \\
\hline

\end{tabular}

\section{Requirements}

\subsection{Functional Requirements}

\begin{enumerate}

\item Configuration
  \begin{itemize}
  \item instantiation, administration
  \item topology coordination
  \item metric collection spec.
  \end{itemize}

\item Metric Collection
  \begin{itemize}
  \item dCAMP API on top of CAMP
  \item filters (at various levels), thresholds
  \item aggregation of metrics across nodes
  \item output to log file
  \end{itemize}

\item Fault Tolerance
  \begin{itemize}
  \item simple rules to handle failures
  \end{itemize}

\end{enumerate}

\subsection{Fault Tolerance}

\begin{enumerate}

\item Topology MUST sustain brief network disconnectivity of any node.

\item Sensor nodes MUST be allowed to enter/exit topology at any time.

\item Root/Collector nodes (i.e. parents) MUST failover in case of extended disconnectivity.
  \begin{itemize}
  \item Loss of previously collected data SHOULD be minimized during failover.
  \end{itemize}

\item Management node SHOULD be allowed to enter/exit topology at any time.

\end{enumerate}

\subsection{Testing}

\begin{itemize}

\item \textbf{Transparency:} dCAMP SHOULD introduce negligible performance impact on sensor nodes.

\item \textbf{Accuracy:} dCAMP MUST accurately report metrics/performance of sensor nodes (individual and aggregated).

\item \textbf{Scalability:} dCAMP SHOULD maintain its transparency and accuracy as it scales (i.e. the number of sensor nodes
increases).

\item \textbf{Fault Tolerance:} dCAMP MUST successfully handle entrance/exit of any node(s) in the system.

\end{itemize}

\section{Execution Steps}

These steps are for the current, prototype version of \dcamp. \textit{Setup: All nodes are running as base nodes
(listening for commands via CLI and network port)}

\begin{enumerate}

\item User initializes root node via CLI (providing configuration) 
\item Root node registers each listening node (configuring as metric nodes) 
\item Metrics are sampled, reported, and logged (repeating indefinitely) 
\item Metric nodes sample metrics and report metrics back to root node 
\item Root node logs received metrics to disk 
\item User shuts down root node via CLI 
\item Root node unregisters each listening node (reverting to base nodes) 
\item Root node terminates (reverting to base node)

\end{enumerate}

\section{ZeroMQ Protocols}

For a quick background on ZeroMQ socket types and message patterns, please see \hyperref[zeromq_primer]{ZeroMQ Primer}.

\subsection{Topology Discovery Protocol}
\label{proto_topo}

The \dcamp distributed topology is dynamically established as the Root node sends out its discovery message and receives
join messages from Base nodes. Once a Base node responds to the Root, the Base node is given its assignment.

\begin{figure}[ht]
    \centering
    \includegraphics[scale=0.66]{topo.png}
    \label{fig:proto_topo_image}
    \caption{Topology Discovery Protocol Diagram}
\end{figure}

\begin{figure}[ht]
\vspace{+10pt}
\begin{verbatim}
topo-discovery = *discover join
discover       = R-MARCO
join           = B-POLO ( R-ASSIGN / R-WTF )
\end{verbatim}
\vspace{-5pt}
\caption[Topology Discovery Protocol]
        {Topology Discovery Protocol: \texttt{R-} represents the Root node sending a message and \texttt{B-}
         represents a Base node sending a message.}
\label{fig:proto_topo_spec}
\end{figure}

\subsubsection{Message Definitions}

\textbf{MARCO} \\
discovery message, two frames

\begin{verbatim}
Frame 0: root endpoint identity, as 0MQ string
Frame 1: root endpoint UUID, 16 bytes in network order
\end{verbatim}

\textbf{POLO} \\
join message, two frames

\begin{verbatim}
Frame 0: base endpoint identity, as 0MQ string
Frame 1: base endpoint UUID, 16 bytes in network order
\end{verbatim}

\textbf{CONTROL} \\
control message, two frames; frame 1 contains the specific topology instructions (level-one collector, leaf node, etc.)

\begin{verbatim}
Frame 0: command, as 0MQ string
Frame 1: properties, JSON-encoded, as 0MQ string
command     = "assignment" / "stop"
properties  = *( parent / level / group / config-file )
parent      = "parent=" <node-endpoint-identity>
level       = "level=" ( "root" / "branch" / "leaf" )
group       = "group=" <group-identity> / NULL
config-file = "config-file=" <file-path>
\end{verbatim}

\textbf{WTF} \\
error message, three frames (the error string may be empty)

\begin{verbatim}
Frame 0: "WTF"
Frame 1: error code, 4 bytes in network order
Frame 2: error message, as 0MQ string
\end{verbatim}

\subsection{Configuration Replication Protocol}
\label{proto_config}

The \dcamp configuration distribution algorithm adheres to the Clustered Hashmap Protocol\cite{chp} with a few minor
(and one major) modifications:

\begin{figure}[ht]
    \centering
    \includegraphics[scale=0.66]{config.png}
    \caption{Configuration Protocol Diagram}
    \label{fig:proto_config_image}
\end{figure}

\begin{enumerate}
\item only the Root node may write new configuration changes,
\item the full configuration table will be replicated across all first-level Collector nodes (lower-level nodes may
      filter their configuration to only store relevant data),
\item a different set of command names are used (as described below), and
\item configuration updates are distributed via the \dcamp hierarchy (instead of directly from the Root node).
\end{enumerate}

A newly assigned first-level Collector node will first subscribe to new configuration updates from the Root node and
then send a configuration snapshot request to the Root node. A newly assigned Sensor (or non-first-level Collector) node
will first subscribe to new configuration updates from its parent Collector node, and then send its parent Collector
node a filtered configuration snapshot request. Once its snapshot has been successfully received, a node will process
any pending configuration updates and then, in the case of a Collector node, respond to child node snapshot requests.

\begin{figure}[ht]
\vspace{+10pt}
\begin{verbatim}
config-distribution = *update / snap-sync
update              = P-KVPUB / P-HUGZ
snap-sync           = C-ICANHAZ *P-KVSYNC ( P-KTHXBAI / P-WTF )
\end{verbatim}
\vspace{-5pt}
\caption[Configuration Protocol Specification]
	{Configuration Protocol Specification: \texttt{P-} represents the parent node (Root or Collector) sending a
	 message and \texttt{C-} represents the child node sending a message.}
\label{fig:proto_config_spec}
\end{figure}

\subsubsection{Message Definitions}

These messages come from the CHP protocol. Additionally, a WTF error message may be sent by the parent in case of error.
It should be noted, each of the following messages is really the same five-frame format with varying key values and
semantics.

\textbf{ICANHAZ} \\
configuration snapshot request

\begin{verbatim}
Frame 0: "ICANHAZ"
Frame 1: sequence number, 8 bytes in network order
Frame 2: <empty>
Frame 3: <empty>
Frame 4: subtree specification
\end{verbatim}

\textbf{KVSYNC} \\
snapshot sync message

\begin{verbatim}
Frame 0: key, as 0MQ string
Frame 1: sequence number, 8 bytes in network order
Frame 2: <empty>
Frame 3: <empty>
Frame 4: value, as blob
\end{verbatim}

\textbf{KTHXBAI} \\
end of successful snapshot sync

\begin{verbatim}
Frame 0: "KTHXBAI"
Frame 1: sequence number, 8 bytes in network order
Frame 2: <empty>
Frame 3: <empty>
Frame 4: subtree specification
\end{verbatim}

\textbf{KVPUB} \\
configuration update sent from parent to child

\begin{verbatim}
Frame 0: key, as 0MQ string
Frame 1: sequence number, 8 bytes in network order
Frame 2: UUID, 16 bytes in network order
Frame 3: properties, JSON-encoded, as 0MQ string
Frame 4: value, as blob

properties = ...
\end{verbatim}

\textbf{HUGZ} \\
heartbeat message sent from parent to child (when no config updates)

\begin{verbatim}
Frame 0: "HUGZ"
Frame 1: 00000000
Frame 2: <empty>
Frame 3: <empty>
Frame 4: <empty>
\end{verbatim}

\subsection{Data Flow Protocol}
\label{proto_data}

There are two data flow protocols in the \dcamp system: the external protocol for data flowing from one node to the next
(via PUB/SUB) and the internal protocol for data flowing between components of a single node (via PUSH/PULL). Both
protocols have the same specification and use the same message formats.

\begin{figure}[ht]
    \centering
    \includegraphics[scale=0.66]{data.png}
    \caption{Data Flow Diagram}
    \label{fig:proto_data_image}
\end{figure}

The \dcamp data flow protocol is very simple, comprised of a single data message type and a heartbeat message type. The
data flows from one node to another via PUB/SUB sockets. Internally, data flows from the upstream data producers,
through a filtering/processing unit, and out to downstream data consumers.

When data rate is slower than a predefined threshold, heartbeats are sent instead to keep inter-node connections alive.

\begin{figure}[ht]
\vspace{+10pt}
\begin{verbatim}
data-flow = *( METRIC / HUGZ )
\end{verbatim}
\vspace{-5pt}
\caption[Data Flow Specification]
	{Data Flow Specification: All messages are sent from child (Metric or Collector) to parent (Collector or Root).}
\label{fig:proto_data_spec}
\end{figure}

\subsubsection{Metric Types}

A metric message must contain: source (node or aggregation), type (e.g. basic, sum, etc.), time (or range), value(s)
(possibly numerator and denominator). Should these types be raw until collected by the root node?

\begin{itemize}
\item basic -- v1 = value at t1, v2 = empty, t2 = empty
\item sum -- v1 = sum at t1, v2 = empty, t2 = empty
\item average -- v1 = sum between t1 and t2, v2 = count
\item percent -- v1 = numerator at t1, v2 = denominator at t1, t2 = empty
\item rate -- v1 = value at t1, v2 = value at t2
\end{itemize}

NOTE ABOUT ACCUMULATION (DOESN'T WORK)

\begin{itemize}
\item filtering can be done two ways: accumulatively and discretely.
\item accumulative means we send only one final value for each time range (e.g. collect every second but report every
      minute, so 60 samples are combined into a single value and sent)
\item discrete means we send each constituent value for each time range, but they are "held" until the time limit is
      reached
\item also, how does this interact with value-based limits? these are always discrete?
\item ACTUALLY: accumulation is not valuable for monotonically increasing values--it is the same as just sampling at the
      slower frequency. accumulation is only valuable for non-monotonically increasing values. but in that case, one
      should find the raw, monotonically increasing values from which it is calculated.
\end{itemize}

\subsubsection{Metric Extensions}

\textbf{VARIABLE LENGTH DATA}
Does \dcamp need to support arbitrary data lengths?

\textbf{NESTED METRICS}
It could be possible for data messages to contain nested data messages, e.g. average/sum of rates.

\textbf{GROUPINGS}
\dcamp may need a more compact data message format for combining multiple metrics into a single message, e.g. for
aggregation purposes or representing entire branches in the topology.
group by source/type/time?
group start/end frames?

\textbf{HISTOGRAMS}
\dcamp should provide a metric histogram ability. Perhaps this should done at each node or only at the root.

\textbf{METRIC REQUESTS}
The admin can use \dcamp to request metrics on a one-time basis, for example, to enumerate the available disks on each
node. The key here is a metric collection is not based on configuration file but rather on real-time input from the
end-user.

Perhaps there should be some special, e.g. "once", metric collection specifications so config data can be sent up to the
root only at node start.

\subsubsection{Message Definitions}

\textbf{METRIC} \\
message containing performance metric data; value v2 will be empty for basic and sum metric types; time t2 will NOT be
empty for average and rate metric types; in case of HUGZ message type, no other property strings are used and frames 3 -
5 are all empty. The config property is only present for internal messages (sent from Sensor to Filter) and represents
the metric configuration's unique name.

\begin{verbatim}
Frame 0: data source (leaf or collector node endpoint), as 0MQ string
Frame 1: properties, JSON-encoded, as 0MQ string
Frame 2: time t1 in ms epoch utc, 8 bytes in network order
Frame 3: value v1, 8 bytes in network order
Frame 4: time t2 in ms epoch utc, 8 bytes in network order; not empty for average and rate
Frame 5: value v2, 8 bytes in network order; empty for basic and sum

properties = *( type / detail / config )
type       = "type=" ( "HUGZ" / "basic" / "sum" / "average" / "percent" / "rate" )
detail     = "detail=" <string>
config     = "config-name=" <string>
\end{verbatim}

% \input{proto_elec}
