\chapter{Related Work}
\label{related_work}

Being distributed, a framework must collect data from a large number of nodes and aggregate the data to one node or
client. Implementations have been built using centralized, hierarchical, peer-to-peer and any number of other
architectures. There are three types of metric gathering techniques: (1) hardware counters and sensors use specialized
hardware to gather highly accurate metrics and are highly dependent on the underlying hardware architecture, (2)
software sensors use modern operating system interfaces to acquire moderately accurate performance metrics in an
architecture-independent interface, and (3) hybrid approaches use a combination of hardware and software sensors to
attain a balance between the two.

There are a number of distributed performance frameworks being actively researched and developed, both academically and
commercially. The frameworks listed in this section were chosen based on their categorization in the
\cite{zanikolas2005} taxonomy; only level 2 frameworks are included. Level 2 frameworks are defined as having at least
one type of republisher in addition to producers; these frameworks usually distribute functionality across multiple
hosts. \cite{zanikolas2005} A limited analysis is conducted by reviewing the available literature, and further analysis
(i.e., verifying scalability, transparency, and validity) is left as future work.

\subsubsection{NetLogger}

Work done by Brian Tierney and Dan Gunter \cite{tierney1998} \cite{gunter2000} presents the Network Application Logger
Toolkit (NetLogger). This framework can be used to monitor the performance of distributed systems at a very detailed
level. With a new logging format and activation service \cite{gunter2002} the authors improved upon their previous work
and increased the toolkit's scalability and data delivery models. NetLogger is being actively developed and is one of
the more well known distributed performance frameworks. The toolkit is composed of four parts: an API and library for
instrumenting a given application, a set of tools for collecting and sorting logs, performance sensors, and a
visualization user interface for the log files.

Each part assumes the system clocks of the individual nodes are accurate and synchronized (the authors mention the use
of NTP to achieve a required clock synchronization of one millisecond). The instrumentation of code allows NetLogger to
gather more detailed data from an application-to-application communication path, such as traces of network packets
through a call hierarchy. The instrumentation also allows the activation service to update the monitoring of parts of
the system dynamically as consumers subscribe to various events and metrics.

Their research has shown NetLogger to be highly scalable, complete, and transparent as well as valid. The activation
service provides a push data delivery and can utilize the security mechanisms part of current web services in order to
authenticate requests for performance data. NetLogger is currently implemented for C, C++, Java, Perl, and Python
applications. Because the framework lacks black box characteristics, its portability is greatly reduced.

\subsubsection{JAMM}

Java Agents for Monitoring and Management (JAMM) \cite{tierney2000} is the fruit of work by the authors of NetLogger to
build a monitoring system with managed sensors. 

The JAMM system consists of six components: sensors, sensor managers, event gateways, directory service, event
consumers, and event archives. There is a sensor manager on each host, with the sensors acting as producers for the
gateways which they publish the data to. The gateways can then filter and aggregate the incoming data according to
consumer queries. The directory service is used to publish the location of the sensors and gateways, allowing for
dynamic discovery of active sensors by the consumers. The event archive is used for historical analysis purposes.

JAMM explicitly uses a pull data delivery model where data is only sent when requested by a consumer. The overall
architecture is generally distributed with the directory service being centralized. JAMM, being heavily based off of
NetLogger, inherits the validity, completeness, security, and transparency of NetLogger along with its lack of
portability. JAMM does, however, prove itself in terms of scalability with it's own architecture.

\subsubsection{Hawkeye}

Hawkeye \cite{hawkeye} is a monitoring and management tool for distributed systems which makes use of technology
previously researched and developed as part of the Condor project \cite{litzkow1988}. Condor provides mechanisms for
collecting information about large distributed computer systems. Hawkeye is being readily developed and is freely
available for download on Linux and Solaris.

Hawkeye uses a general push delivery model by configuring Condor to execute programs, or modules, at given time
intervals, collect performance data, and send it to the central manager. These modules are configurable such that the
``period'' of module execution can be set to a given time frame in seconds, minutes, or hours or the module can be
executed in ``continuous'' mode where the module's execution never ends. The available modules for monitoring a Condor
pool include: disk space, memory used, network errors, open files, CPU monitoring, system load, users, Condor Node,
Condor Pool, and Grid Probe. Custom modules can also be developed and installed for monitoring of arbitrary resources
and metrics. Data can be accessed from the central manager via an API, CLI, or GUI.

While no experiments have been run, the generally centralized manager reduces the Hawkeye framework's scalability, and
its transparency is unknown. The frameworks module based producer architecture gives it an infinite completeness, but
being only available on Linux and Solaris makes the framework less portable. Lastly, the ability to run jobs securely on
target machines has been left as future work by the authors.

\subsubsection{SCALEA-G}

Truong and Fahringer present SCALEA-G \cite{truong2004}, an unified monitoring and performance analysis system for
distributed systems. It is based on the Open Grid Service Architecture \cite{foster2002} and allows for a number of
services to monitor both grid resources and grid application. SCALEA-G uses dynamic instrumentation to profile and trace
Java and C/C++ applications in both push and pull data delivery models, making the framework both scalable and portable.

The SCALEA-G framework is composed of several services: directory service, archival service, sensor manager service,
instrumentation service, client service, and user portal. These services provide the following functionality
respectively: publishing and searching of producers and consumers, storage of performance results, management of
sensors, dynamic instrumentation of source code, administering clients and analyzing data, and on-line monitoring and
performance analysis.

The framework makes use of secure sockets to achieve secure communications and achieves high completeness via code
instrumentation. Unfortunately, the authors do not provide any report on SCALEA-G's validity or transparency.

\subsubsection{IMPuLSE}

Integrated Monitoring and Profiling for Large Scale Environments \cite{bridges2004} was designed to address ``operating
system-induced performance anomalies'' and provide ``accurate, low-overhead, whole-system monitoring.'' The authors have
chosen to develop a message-centric approach which associates data with messages rather than hosts and a system-wide
statistical sampling to increases the framework's scalability.

The IMPuLSE framework is still in the design stage, and therefore lacks any implementation data outside of their new
message-centric design pattern which shows promising results. Unfortunately, this leaves the framework with unknown
transparency, security, completeness, portability, and validity.

\subsubsection{Host sFlow}

sFlow is a network monitoring protocol consisting of sFlow Agents, built directly into the router and switch network
device management layer by each vendor, which analyze traffic and send metrics to sFlow Collectors on the network.
\cite{sflow} Host sFlow is an open-source implementation of the sFlow protocol which uses sFlow Agents to monitor
multi-vendor physical and virtual servers. Host sFlow is capable of application layer monitoring (e.g. node.js,
Memcached), as well, and may be implemented directly by device/OS manufacturers for easier deployment. \cite{host-sflow}

In supporting host and application performance metric analysis alongside network metrics in one common system, sFlow has
an advantage over more traditional host-only distributed performance frameworks. While sFlow's claims to scalable and
accurate network level monitoring have been validated \cite{needed}, less work has been to show the same for Host sFlow.
\cite{needed}

\subsubsection{Ganglia}

Ganglia is a distributed performance framework designed specifically for high-performance computing (HPC) environments,
and it has been used to monitor real-world HPC, grid, and ``planetary-scale'' systems. Ganglia uses different protocols
for intra- and inter-cluster communication: a multicast listen/announce protocol within a single cluster and a tree of
point-to-point connections between clusters. Ganglia is well used and is actively used to monitor over 500 different
systems. \cite{ganglia}

The analysis presented in \cite{ganglia} shows the design scales and maintains transparency for systems of several
hundred nodes. Still, scalability is a concern of the authors since the multicast protocol exhibits a quadratic
trendline as the number of nodes within a cluster increases. Memory usage and inter-cluster bandwidth also increase as
the number of nodes increases, albeit much more linearly. In comparison, \dcamp memory usage is nearly constant since
performance data is not persisted in memory to the same extent.

\subsubsection{Conclusion}

There are a number of high quality and effective distributed performance frameworks being actively researched and
developed, but with some frameworks having more research than others, there is a natural disparity of information about
each framework. While the frameworks vary in distributed architecture and features, they all fulfill the minimum
requirements of performance frameworks. The frameworks listed in this work are mainly software based sensor frameworks.
This was chosen due to the inherent portability advantage of software sensors over hardware or hybrid sensors.

Many authors have failed to address their framework's validity, transparency, and scalability explicitly, thinking the
framework's architecture speaks for itself or blindly assuming it is accurate and introduces negligible load on the
measured system. I leave it as future work to conduct formal experiments to test validity, transparency, and scalability
of the distributed performance frameworks listed here.
