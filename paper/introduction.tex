\chapter{Introduction}
\label{introduction}

As the Internet has become more pervasive in today's business economy, there has been a natural trend of distributing
large, complex systems across multiple components locally and throughout the world. These systems are not always
homogeneous with respect to hardware architecture or even operating system, and development of these system can prove to
be quite difficult even with the best tools available. In order to effectively build these systems, software engineers
must be able to test their system for performance defects as well as bottlenecks. Additionally, distributed systems must
respond to changes in availability and work load on its individual nodes.

Distributed performance testing frameworks supply software practitioners and system administrators with tools to
evaluate the performance of a system from both black box and white box perspectives by publishing interfaces for
instrumenting, collecting, analyzing, and visualizing performance data across the distributed system and distributed
applications. Distributed performance monitoring frameworks, often considered part of the testing framework, provide a
black box interface into monitoring a distributed system or application and usually includes mechanisms for triggering
actions based on performance events. For the purpose of this work, I introduce the term distributed performance
framework to collectively refer to both distributed performance testing and distributed performance monitoring
frameworks.

\section{Distributed Performance Framework Criterion}

In order for practitioners and researchers alike to effectively choose a distributed performance framework, it is
necessary to have a set criteria for evaluation. Presented here is an extended criterion of the general requirements
presented by \cite{zanikolas2005} for grid systems. Data Delivery Models and Security have been taken directly from
their work. Scalability has been modified to only consider good performance as its goal while Low Intrusiveness has been
turned into Transparency. Extensibility has been removed from the list, and Completeness and Validity have been added.
This work provides an alternate definition for Portability.

\subsubsection{Data Delivery Models}

Monitoring information includes fairly static (e.g., software and hardware configuration of a given node) and dynamic
events (e.g., current processor load, memory), which suggests the use of different measurement policies (e.g., periodic
or on demand). In addition, consumer patterns may vary from sparse interactions to long lived subscriptions for
receiving a constant stream of events. In this regard, the monitoring system must support both pull and push data
delivery models.
\cite{zanikolas2005}

\subsubsection{Security}

Certain scenarios may require a monitoring service to support security services such as access control, single or mutual
authentication of parties, and secure transport of monitoring information.
\cite{zanikolas2005}

\subsubsection{Scalability}

Monitoring systems have to cope efficiently with a growing number of resources, events and users. This scalability can
be achieved as a result of good performance which guarantees that a monitoring system will achieve the needed throughput
within an acceptable response time in a variety of load scenarios.
\cite{zanikolas2005}

\subsubsection{Transparency}

Transparency refers to the lack of impact a distributed performance framework makes on the system being monitored. As
\cite{zanikolas2005} states, it is ``typically measured as a function of host (processor, memory, I/O) and network load
(bandwidth) generated by the collection, processing and distribution of events.'' If a framework lacks transparency it
will fail to allow the underlying distributed system to perform well and will produce inaccurate performance
measurements, thereby reducing its Scalability and destroying its Validity.

\subsubsection{Completeness}

The Completeness of a distributed performance framework refers to the exhaustiveness to which it gathers performance
metrics. At a minimum, a framework must provide interfaces for measuring and aggregating performance data about a
system's processor, memory, disk, and network usage. Several distributed performance frameworks provide further detailed
performance metrics about the given distributed system being monitored, but this is usually at the cost of Portability.

\subsubsection{Validity}

A distributed performance framework is only as good as the data is produces; if the sensors or gathering techniques are
inaccurate, then the data is useless at best, misleading at worst. Validity of a framework is achieved when the authors
of a framework provide formal verification of its accuracy.

\subsubsection{Portability}

A framework's ability to run on a completely heterogeneous distributed system without special considerations by the
practitioner is what this work defines as Portability. More specifically, a portable framework has a unified API
regardless of the system architecture, does not restrict itself to applications written in specific programming
languages, and does not require practitioners to manually instrument their application code. This black box
characteristic is vital for a viable distributed performance framework's effectiveness as it allows practitioners to
focus on the performance data and not on a myriad of APIs for various architectures or languages.

\section{\dcamp}
\label{dcamp}

The Distributed Common API for Measuring Performance (\dcampns) is a distributed performance framework built on top of
Mark Gabel and Michael Haungs' 2007 research on \emph{CAMP: a common API for measuring performance} \cite{gabel2007}.
The fundamental functionality of \camp is providing an accurate and ``consistent method for retrieving system
performance data from multiple platforms.''

\dcamp takes advantage of this functionality and the authors' work done in validating \campns's accuracy and adds the
core feature sets listed below. As shown in the analysis work presented in Chapter \ref{analysis}, \dcamp adds these
features while still maintaining minimal impact on the systems, processes, and networks being monitored.

The key contributions of the Distributed Common API for Measuring Performance are:

\begin{itemize}
\item a stateful performance API,
\item distributed performance data aggregation,
\item performance filters and triggers, and
\item simplistic fault tolerance.
\end{itemize}

\subsection{Terminology}
Knowing the following terminology will make it easier to understand and discuss the \dcamp project, its main goals, its
usage, its components, and its inner workings. 

\begin{description}

\item[Distributed Performance Testing Framework (DPTF),]
\item[Distributed Performance Monitoring Framework (DPMF),]
\item[Distributed Performance Framework (DPF):]

An DPTF or DPMF (collectively termed DPF) is a framework which allows its users to evaluate the performance of a system
from both black box and white box perspectives by publishing interfaces for instrumenting, collecting, analyzing, and
visualizing performance data across the distributed system and distributed applications. Typically, the framework
provides a black box interface into monitoring a distributed system or application and includes mechanisms for
triggering actions based on performance events. The \dcamp project is designed to be a DPF. 

\item[Performance Metric,]
\item[Performance Counter:]
Performance metrics are any data about a given node relating to its throughput, capacity, utilization, or latency. In
\dcampns, these are grouped into four different sets of performance metrics---global, network, disk, and
per-process---and a fifth set of inquiry metrics. They are described fully in section \ref{dcamp_metrics}. 

\item[Metric Aggregation:]
Metric aggregation is the process of combining metrics from multiple nodes into a single metric. Performance metrics,
while useful at an individual system granularity, can be rather limited in value for a DPF where the goal is measurement
of the distributed system as a whole. Metric aggregation provides a coarser granularity for the performance metrics,
calculating a sum, average, percent, or any other mathematically relevant operation across multiple nodes in the system. 

\item[Metric Calculation:]
Metric calculation is the process of combining identical metrics from multiple timestamps into a single metric. Various
equations and inputs are used to do this calculation, chosen depending on the type of metric and desired representation;
these equations are listed in Table \ref{tab:metric_types}.

\item[Filter,]
\item[Throttle,]
\item[Threshold:]
Filtering (or throttling or thresholding) provides a mechanism for reducing the amount of data sent between nodes of the
system. Filtering allows a user to specify when or at what point to report metrics from one level to its parent. For
example, a filter might be set to only report average CPU utilization that is over seventy-five percent. 

\item[\dcamp Node,]
\item[\dcamp Process:]
A single, independently running instance of \dcamp in the distributed system is called a \dcamp node or process. More
than one node may exist on a single computer. A node consists of the Node role and zero or more other \dcamp roles. 

\item[\dcamp Service:]
Services are a way of logically grouping functions within the \dcamp system, from performance metric sampling to \dcamp
system management. A description of all the \dcamp services can be found in section \ref{roles_and_services}. Each
service is implemented in \dcamp as an independent thread.

\item[\dcamp Role:]
Roles in the \dcamp system are groupings of one or more \dcamp services. There does not exist a one-to-one
relationship between roles and services; the \dcamp role-to-service mapping can be seen in Table
\ref{tab:role_to_services}.

\item[\dcamp Hierarchy:]
The \dcamp system is organized in a hierarchical pattern with respect to data movement and system control functionality.
The hierarchy can be thought of as a tree structure, with leaf nodes being at the top of the hierarchy and a single root
node at the bottom. Metric data moves down the hierarchy from leaves to the root; configuration data and control
commands move up from the root to the leaves. 

\item[\dcamp Level:]
Levels are a way of organizing the \dcamp hierarchy horizontally. Levels are defined by their distance from the root
node. For example, level one is one node away from the root node, or said another way, the first level is directly
``connected'' to the root node. The second level is two nodes away from the root node, or any node in the second level is
connected to the root node by another node (in the first level). This necessarily means the root is in level zero (all
by itself).

\item[Parent Node:]
Nodes are called parent nodes if there exists at least one node connected to it from a level of higher ordinal value.
For example, a node in level one with at least one node connected to it from level two is considered a parent node. The
root node is inherently a parent node. 

\item[Child Node:]
A node is called a child node if it is connected to another node in a level of lower ordinal value. For example, a node
in level one is connected to the root node (in level zero), so it is called a child node. The root node is the only
node in the \dcamp system which is not a child node. 

\item[\dcamp Configuration:]
The \dcamp configuration specifies everything about the system, including hierarchy levels, metrics, sampling periods,
reporting periods, filtering, communication details, etc. The configuration is set at the root node and then
distributed to the rest of the \dcamp system. Configuration details can be found in section \ref{configuration}. 

\item[ZeroMQ,]
\item[zmq,]
\item[\O MQ:]
ZeroMQ is a message queuing framework which allows a developer to build distributed systems by focusing on the data
and implementing simple design patterns.

\item[ZeroMQ Address:]
A \O MQ address is the combination of network host identifier (i.e. an IP Address or resolvable name) and Internet
socket port number. 

\item[ZeroMQ Endpoint:]
An endpoint is the combination of any ZeroMQ transport (\texttt{pgm}, \texttt{inproc}, \texttt{ipc}, or \texttt{tcp})
and a \O MQ address.

\item[Metric Collection,]
\item[Metric Sampling:]
Metric collection or sampling is the process of measuring metrics on a given node. 

\item[Metric Reporting:]
Metric reporting is the process of sending sampled metrics to a parent node.

\end{description}

