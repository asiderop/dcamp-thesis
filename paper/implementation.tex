\chapter{Implementation}
\label{implementation}

\section{\dcamp Operation}

\subsection{Sequence of \dcamp Operation}
\label{operation_sequnce}

The following steps describe how the \dcamp system is turned on. The \textit{Base} nodes (other than the node assigned
to be the \textit{Root}) can be started at any time by using the \dcamp CLI, before or after the \textit{Root} node is
initialized. It is expected these \textit{Base} nodes are managed by a watchdog utility which automatically restarts the
node if it exits for any reason.

\begin{figure}[H]
\vspace{+10pt}
\begin{lstlisting}[language=bash,frame=single,basicstyle=\footnotesize\ttfamily]
#!/usr/bin/env bash
while [ true ]
do
    dcamp base --address localhost:56789
done
\end{lstlisting}
\vspace{-10pt}
\caption{Sample Watchdog Script}
\label{fig:sample_watchdog}
\end{figure}

\begin{enumerate}

\item User promotes a \textit{Root} node via the \dcamp CLI, specifying a configuration file and a \textit{Base} node's
      address.
\item \textit{Root} node connects to each \textit{Base} node and begins the ``discover'' \hyperref[proto_topo]{Topology
      Protocol}.
\item \textit{Base} nodes join the \dcamp system at any time, being assigned as \textit{Collector} or \textit{Metric}
      nodes in the topology.

\item \dcamp runs in a steady state, nodes entering or exiting the system at any time.

      \begin{itemize}
      \item Performance counters are sampled, filtered, reported, and logged by the Metric nodes at regular intervals
            according to the \hyperref[configuration]{\dcamp Configuration}.
      \item Performance counters received from child nodes are aggregated, filtered, reported, and logged by
            \textit{Collector} nodes at regular intervals according to the \dcamp Configuration.
      \item Performance counters received from child nodes are aggregated and logged by \textit{Root} node for later
	    processing (e.g. graphing metrics during a test scenario or correlating statistics with a distributed event
	    log).
      \end{itemize}

\item User stops \dcamp by using the \dcamp CLI command.
\item \textit{Root} node begins the ``stop'' Topology Protocol.
\item \textit{Collector} and \textit{Metric} nodes exit the topology and revert to \textit{Base} nodes.
\item \textit{Root} node exits, reverting to \textit{Base} node.

\end{enumerate}

\subsection{Threading Model}
\label{threading_model}

As mentioned above as the first and third steps of \dcamp operation, a \textit{Base} node can transform into one of the
three active \dcamp roles: \textit{Root}, \textit{Collector}, or \textit{Metric}. This transformation is actually the
\textit{Base} role (via the Node service) launching and managing another role internally. This interaction is depicted
in Figure \ref{fig:node_role_service_image}.

\begin{figure}[H]
    \centering
    \includegraphics[scale=0.5]{node-role-service.pdf}
    \caption[Node, Role, Services Threading Model Diagram]
            {Node, Role, Services Threading Model Diagram: Thread boundaries are represented by dashed lines. Except for
	     the Node service's \texttt{SUB} and \texttt{REQ} sockets, all arrows represent \texttt{PAIR} socket
	     communication.}
    \label{fig:node_role_service_image}
\end{figure}

When a \textit{Base} node is running, only the bottom two threads (the \textit{Base} role and the Node service) are
active. Once it receives an assignment from the ``discover'' Topology Protocol or the \dcamp CLI, the Node service
launches an appropriate role thread which, in turn, launches one or more role-specific service threads.

All communication between the roles and services occurs across \texttt{PAIR} control sockets. There are also various
service-to-service communications which occur via \texttt{inproc} transport sockets (e.g. the internal
\hyperref[proto_data]{Data Flow Protocol}) and shared memory data structures (e.g. the Configuration service).

Also mentioned in section \ref{operation_sequnce} as the last two steps, each role exits and, by doing so, reverts
itself back to a \textit{Base} node. This is handled just like before, with the Node service receiving a \texttt{STOP}
message via the ``stop'' Topology Protocol and then notifying the internally running role to shut down. The role thread
then notifies its service threads, waits for them to finish, then exits.

\section{ZeroMQ Protocols}

ZeroMQ is a fantastic message queuing framework that essentially provides more intelligent sockets as building blocks
for distributed systems. ZeroMQ handles the intricacies of sending messages between two endpoints and lets the
application handle the rest of the logic. The protocols described in this section do not come \textit{from} ZeroMQ,
rather they are built \textit{using} ZeroMQ sockets and message patters.

For a quick background on ZeroMQ socket types and message patterns, please see Appendix \ref{zeromq_primer}.

\subsection{Topology Discovery Protocol}
\label{proto_topo}

The \dcamp distributed topology is dynamically established as the Root node sends out its discovery message and receives
join messages from Base nodes. Once a Base node responds to the Root, the Base node is given its assignment.

\begin{figure}[ht]
    \centering
    \includegraphics[scale=0.66]{topo.png}
    \label{fig:proto_topo_image}
    \caption{Topology Discovery Protocol Diagram}
\end{figure}

\begin{figure}[ht]
\vspace{+10pt}
\begin{verbatim}
topo-discovery = *discover join
discover       = R-MARCO
join           = B-POLO ( R-ASSIGN / R-WTF )
\end{verbatim}
\vspace{-5pt}
\caption[Topology Discovery Protocol]
        {Topology Discovery Protocol: \texttt{R-} represents the Root node sending a message and \texttt{B-}
         represents a Base node sending a message.}
\label{fig:proto_topo_spec}
\end{figure}

\subsubsection{Message Definitions}

\textbf{MARCO} \\
discovery message, two frames

\begin{verbatim}
Frame 0: root endpoint identity, as 0MQ string
Frame 1: root endpoint UUID, 16 bytes in network order
\end{verbatim}

\textbf{POLO} \\
join message, two frames

\begin{verbatim}
Frame 0: base endpoint identity, as 0MQ string
Frame 1: base endpoint UUID, 16 bytes in network order
\end{verbatim}

\textbf{CONTROL} \\
control message, two frames; frame 1 contains the specific topology instructions (level-one collector, leaf node, etc.)

\begin{verbatim}
Frame 0: command, as 0MQ string
Frame 1: properties, JSON-encoded, as 0MQ string
command     = "assignment" / "stop"
properties  = *( parent / level / group / config-file )
parent      = "parent=" <node-endpoint-identity>
level       = "level=" ( "root" / "branch" / "leaf" )
group       = "group=" <group-identity> / NULL
config-file = "config-file=" <file-path>
\end{verbatim}

\textbf{WTF} \\
error message, three frames (the error string may be empty)

\begin{verbatim}
Frame 0: "WTF"
Frame 1: error code, 4 bytes in network order
Frame 2: error message, as 0MQ string
\end{verbatim}

\subsection{Configuration Replication Protocol}
\label{proto_config}

The \dcamp configuration distribution algorithm adheres to the Clustered Hashmap Protocol\cite{chp} with a few minor
(and one major) modifications:

\begin{figure}[ht]
    \centering
    \includegraphics[scale=0.66]{config.png}
    \caption{Configuration Protocol Diagram}
    \label{fig:proto_config_image}
\end{figure}

\begin{enumerate}
\item only the Root node may write new configuration changes,
\item the full configuration table will be replicated across all first-level Collector nodes (lower-level nodes may
      filter their configuration to only store relevant data),
\item a different set of command names are used (as described below), and
\item configuration updates are distributed via the \dcamp hierarchy (instead of directly from the Root node).
\end{enumerate}

A newly assigned first-level Collector node will first subscribe to new configuration updates from the Root node and
then send a configuration snapshot request to the Root node. A newly assigned Sensor (or non-first-level Collector) node
will first subscribe to new configuration updates from its parent Collector node, and then send its parent Collector
node a filtered configuration snapshot request. Once its snapshot has been successfully received, a node will process
any pending configuration updates and then, in the case of a Collector node, respond to child node snapshot requests.

\begin{figure}[ht]
\vspace{+10pt}
\begin{verbatim}
config-distribution = *update / snap-sync
update              = P-KVPUB / P-HUGZ
snap-sync           = C-ICANHAZ *P-KVSYNC ( P-KTHXBAI / P-WTF )
\end{verbatim}
\vspace{-5pt}
\caption[Configuration Protocol Specification]
	{Configuration Protocol Specification: \texttt{P-} represents the parent node (Root or Collector) sending a
	 message and \texttt{C-} represents the child node sending a message.}
\label{fig:proto_config_spec}
\end{figure}

\subsubsection{Message Definitions}

These messages come from the CHP protocol. Additionally, a WTF error message may be sent by the parent in case of error.
It should be noted, each of the following messages is really the same five-frame format with varying key values and
semantics.

\textbf{ICANHAZ} \\
configuration snapshot request

\begin{verbatim}
Frame 0: "ICANHAZ"
Frame 1: sequence number, 8 bytes in network order
Frame 2: <empty>
Frame 3: <empty>
Frame 4: subtree specification
\end{verbatim}

\textbf{KVSYNC} \\
snapshot sync message

\begin{verbatim}
Frame 0: key, as 0MQ string
Frame 1: sequence number, 8 bytes in network order
Frame 2: <empty>
Frame 3: <empty>
Frame 4: value, as blob
\end{verbatim}

\textbf{KTHXBAI} \\
end of successful snapshot sync

\begin{verbatim}
Frame 0: "KTHXBAI"
Frame 1: sequence number, 8 bytes in network order
Frame 2: <empty>
Frame 3: <empty>
Frame 4: subtree specification
\end{verbatim}

\textbf{KVPUB} \\
configuration update sent from parent to child

\begin{verbatim}
Frame 0: key, as 0MQ string
Frame 1: sequence number, 8 bytes in network order
Frame 2: UUID, 16 bytes in network order
Frame 3: properties, JSON-encoded, as 0MQ string
Frame 4: value, as blob

properties = ...
\end{verbatim}

\textbf{HUGZ} \\
heartbeat message sent from parent to child (when no config updates)

\begin{verbatim}
Frame 0: "HUGZ"
Frame 1: 00000000
Frame 2: <empty>
Frame 3: <empty>
Frame 4: <empty>
\end{verbatim}

\subsection{Data Flow Protocol}
\label{proto_data}

There are two data flow protocols in the \dcamp system: the external protocol for data flowing from one node to the next
(via PUB/SUB) and the internal protocol for data flowing between components of a single node (via PUSH/PULL). Both
protocols have the same specification and use the same message formats.

\begin{figure}[ht]
    \centering
    \includegraphics[scale=0.66]{data.png}
    \caption{Data Flow Diagram}
    \label{fig:proto_data_image}
\end{figure}

The \dcamp data flow protocol is very simple, comprised of a single data message type and a heartbeat message type. The
data flows from one node to another via PUB/SUB sockets. Internally, data flows from the upstream data producers,
through a filtering/processing unit, and out to downstream data consumers.

When data rate is slower than a predefined threshold, heartbeats are sent instead to keep inter-node connections alive.

\begin{figure}[ht]
\vspace{+10pt}
\begin{verbatim}
data-flow = *( METRIC / HUGZ )
\end{verbatim}
\vspace{-5pt}
\caption[Data Flow Specification]
	{Data Flow Specification: All messages are sent from child (Metric or Collector) to parent (Collector or Root).}
\label{fig:proto_data_spec}
\end{figure}

\subsubsection{Metric Types}

A metric message must contain: source (node or aggregation), type (e.g. basic, sum, etc.), time (or range), value(s)
(possibly numerator and denominator). Should these types be raw until collected by the root node?

\begin{itemize}
\item basic -- v1 = value at t1, v2 = empty, t2 = empty
\item sum -- v1 = sum at t1, v2 = empty, t2 = empty
\item average -- v1 = sum between t1 and t2, v2 = count
\item percent -- v1 = numerator at t1, v2 = denominator at t1, t2 = empty
\item rate -- v1 = value at t1, v2 = value at t2
\end{itemize}

NOTE ABOUT ACCUMULATION (DOESN'T WORK)

\begin{itemize}
\item filtering can be done two ways: accumulatively and discretely.
\item accumulative means we send only one final value for each time range (e.g. collect every second but report every
      minute, so 60 samples are combined into a single value and sent)
\item discrete means we send each constituent value for each time range, but they are "held" until the time limit is
      reached
\item also, how does this interact with value-based limits? these are always discrete?
\item ACTUALLY: accumulation is not valuable for monotonically increasing values--it is the same as just sampling at the
      slower frequency. accumulation is only valuable for non-monotonically increasing values. but in that case, one
      should find the raw, monotonically increasing values from which it is calculated.
\end{itemize}

\subsubsection{Metric Extensions}

\textbf{VARIABLE LENGTH DATA}
Does \dcamp need to support arbitrary data lengths?

\textbf{NESTED METRICS}
It could be possible for data messages to contain nested data messages, e.g. average/sum of rates.

\textbf{GROUPINGS}
\dcamp may need a more compact data message format for combining multiple metrics into a single message, e.g. for
aggregation purposes or representing entire branches in the topology.
group by source/type/time?
group start/end frames?

\textbf{HISTOGRAMS}
\dcamp should provide a metric histogram ability. Perhaps this should done at each node or only at the root.

\textbf{METRIC REQUESTS}
The admin can use \dcamp to request metrics on a one-time basis, for example, to enumerate the available disks on each
node. The key here is a metric collection is not based on configuration file but rather on real-time input from the
end-user.

Perhaps there should be some special, e.g. "once", metric collection specifications so config data can be sent up to the
root only at node start.

\subsubsection{Message Definitions}

\textbf{METRIC} \\
message containing performance metric data; value v2 will be empty for basic and sum metric types; time t2 will NOT be
empty for average and rate metric types; in case of HUGZ message type, no other property strings are used and frames 3 -
5 are all empty. The config property is only present for internal messages (sent from Sensor to Filter) and represents
the metric configuration's unique name.

\begin{verbatim}
Frame 0: data source (leaf or collector node endpoint), as 0MQ string
Frame 1: properties, JSON-encoded, as 0MQ string
Frame 2: time t1 in ms epoch utc, 8 bytes in network order
Frame 3: value v1, 8 bytes in network order
Frame 4: time t2 in ms epoch utc, 8 bytes in network order; not empty for average and rate
Frame 5: value v2, 8 bytes in network order; empty for basic and sum

properties = *( type / detail / config )
type       = "type=" ( "HUGZ" / "basic" / "sum" / "average" / "percent" / "rate" )
detail     = "detail=" <string>
config     = "config-name=" <string>
\end{verbatim}

\subsection{Recovery Protocols}
\label{proto_reco}

The \dcamp Recovery Protocols are used for the \hyperref[algor_promo]{Promotion} and \hyperref[algor_elect]{Election}
algorithms and use the same base messages as the \hyperref[proto_topo]{Topology Protocol}, \texttt{TOPO} and
\texttt{CONTROL}.

\begin{figure}[H]
\vspace{+10pt}
\begin{verbatim}
branch-recovery  = *sos group-stop
sos              = B-SOS
group-stop       = R-GROUP B-POLO R-STOP
\end{verbatim}
\vspace{-5pt}
\caption[Branch Recovery Protocol]
	{Branch Recovery Protocol: \texttt{R-} represents the Root node sending a message and \texttt{B-} represents a
	 Base node sending a message.}
\label{fig:proto_reco_branch_spec}
\end{figure}

The Branch Recovery Protocol is initiated by Base nodes when they detect their Collector has died. Once the Root node
has received an \texttt{SOS} message from at least one third of the branch's Base nodes, the Root proceeds to shutdown
the entire branch using the Topology Protocol.

The \texttt{GROUP} message is shorthand for the Topology Protocol's \texttt{TOPO} message with a key value of
\texttt{"/GROUP/<group-name>"}. This takes advantage of ZeroMQ's Pub-Sub filtering to only stop the faulty branch.

\begin{figure}[H]
\vspace{+10pt}
\begin{verbatim}
root-recovery  = *election
election       = C-WUTUP *C-YO C-IWIN
\end{verbatim}
\vspace{-5pt}
\caption[Root Recovery Protocol]
        {Root Recovery Protocol: \texttt{C-} represents a Collector node sending a message.}
\label{fig:proto_reco_root_spec}
\end{figure}

As each Collector nodes detect the Root node has died, it attempts to start an election via the Root Recovery Protocol.
The \texttt{WUTUP} and \texttt{IWIN} messages are shorthand for \texttt{TOPO(key="/RECOVERY/wutup"} and
\texttt{TOPO(key="/RECOVERY/iwin"} respectively. The \texttt{YO} message is shorthand for
\texttt{CONTROL(command="yo")}.


