\chapter{ZeroMQ Primer}
\label{zeromq_primer}

\section{Why ZeroMQ}

Not surprisingly, the most succinct description of ZeroMQ is found in \textit{The Guide}\cite{zguide} preface,

\begin{quote}
\O MQ (also known as ZeroMQ, 0MQ, or zmq) looks like an embeddable networking library but acts like a concurrency
framework. It gives you sockets that carry atomic messages across various transports like in-process, inter-process,
TCP, and multicast. You can connect sockets N-to-N with patterns like fan-out, pub-sub, task distribution, and
request-reply. It's fast enough to be the fabric for clustered products. Its asynchronous I/O model gives you scalable
multicore applications, built as asynchronous message-processing tasks. It has a score of language APIs and runs on most
operating systems. \O MQ is from iMatix [\url{http://www.imatix.com}] and is LGPLv3 open source.
\end{quote}

No appendix could justly explain ZeroMQ or give the reader a true understanding of its abilities and proper use. Read
\textit{The Guide} (a freer and more up-to-date version is online at \url{http://zguide.zeromq.org}) and, if truly
adventurous (or just morbidly curious), go through all 750+ examples in any of the 28 programming languages available.

Forget about RPC, MPI, and raw sockets. ZeroMQ allows a developer to build distributed systems by focusing on the data
and implementing simple design patterns. In short, ZeroMQ allows the distributed systems developer to have fun. No joke.

\section{Sockets and Message Patterns}

To begin understanding ZeroMQ, a foundational knowledge of \O MQ sockets, messages, and patterns is needed.

\subsection{Sockets and Messages}

\O MQ sockets mimic standard TCP sockets, exposing interfaces for creating and destroying instances, binding and
connecting to network endpoints, and sending and receiving data. However, they have two key differences from their
TCP counterparts.

First, they are asynchronous---the actual sending and receiving of data on a ZeroMQ socket is handled by a background
thread. Second, \O MQ sockets have built-in support for one-to-many connections. That is, a single socket can send and
receive data from multiple endpoints.

ZeroMQ sockets are explicitly typed, with the type dictating how data is routed and queued to and from the socket.
Furthermore, this explicit typing means only certain socket types can be connected to each other.

ZeroMQ messages are the building blocks of all data sent across ZeroMQ sockets. A message is comprised of one or more
frames (or parts), and a single frame can be any size (including zero) that fits in memory. ZeroMQ guarantees messages
are delivered atomically, meaning either all frames of the message are sent/received or none of the frames. Lastly,
because sockets are asynchronous and messages are atomic, the entire message must fit in memory.

\subsection{Messaging Patterns}

Generally speaking, the ZeroMQ messaging patterns are defined by the socket routing and queuing rules as well as each
socket's compatible type pairings. As listed in the \texttt{zmq\_socket} man page\cite{zmq-socket}, ZeroMQ supports the
following core messaging patterns.

\begin{description}

\item[Publish-Subscribe:]

``The publish-subscribe pattern is used for one-to-many distribution of data from a single publisher to multiple
subscribers in a fan out fashion.''

The two socket types used for this pattern are \texttt{PUB}, which can only send messages, and \texttt{SUB}, which can
only receive messages. Naturally, this is a unidirectional pattern.

\item[Request-Reply:]

``The request-reply pattern is used for sending requests from a [...] client to one or more [...] services, and receiving
subsequent replies to each request sent.''

The two basic socket types for this pattern, \texttt{REQ} and \texttt{REP}, require strict ordering of messages: a
message must be first be sent on the \texttt{REQ} socket before a message can be received on the socket, and vice versa
for the \texttt{REP} socket. Two advanced socket types, \texttt{XREQ} (or \texttt{DEALER}) and \texttt{XREP} (or
\texttt{ROUTER}), allow a more lenient communication pattern.

\item[Pipeline:]

``The pipeline pattern is used for distributing data to nodes arranged in a pipeline. Data always flows down the
pipeline, and each stage of the pipeline is connected to at least one node. When a pipeline stage is connected to
multiple nodes data is round-robined among all connected nodes.''

\texttt{PUSH} (send-only) and \texttt{PULL} (receive-only) socket types are used for this pattern. Like Pub-Sub, this is
a unidirectional pattern.

\item[Exclusive Pair:]

``The exclusive pair pattern is used to connect a peer to precisely one other peer. This pattern is used for inter-thread
communication across the inproc transport.''

Only the \texttt{PAIR} socket type can be used for this pattern.

\end{description}

\section{Useful Features for \dcamp}

Apart from the general happiness ZeroMQ offers to the distributed systems developer, some features are particularly
useful in \dcampns.

\subsection{Topic Filtering}

All messages sent using Pub-Sub are filtered (usually by the publisher) based on the ``topics'' to which each subscriber
subscribes. Topics, stated simply, are the leading bytes of a message's first frame. By default, a \texttt{SUB} socket
is not subscribed to any topics.

Consider a \texttt{PUB} message which contains \texttt{b'/fruit/apple'} as its first frame. A subscriber would receive
this message if it subscribed to \texttt{b'/fruit'} or \texttt{b'/'} or even \texttt{b''} (an empty frame). But if the
subscriber was not subscribed to any topics or only subscribed to the \texttt{b'/plants'} topic, it would not receive
the message. Do note: the topic can be any binary data, not just character data.

Topic filtering fits naturally into the \hyperref[proto_config]{Configuration Replication Protocol} where different
roles only replicate portions of the config and the \hyperref[proto_topo]{Topology Discovery Protocol} where the root
node needs to send commands to a subset of the topology.

\subsection{Easy Message Debugging}

ZeroMQ's atomic multipart message passing lends itself to what \textit{The Guide} calls the ``Cheap or Nasty pattern''
\cite{zguide}. That is, use cheap, easy to write/read, overly verbose messages for infrequent control scenarios and
nasty, compact, highly performant messages for long-lived and frequent data scenarios.

In \dcampns, all control messages follow the cheap pattern, making them easy to debug. But the data messages, which tend
to not need a lot of debugging, are free to be more optimized.

\subsection{Simplified Threading Design}

While much care must still be taken in their use, the inherent properties of ZeroMQ sockets (asynchronous nature,
utilization of send/receive queues, ability to round-robin and fair-queue messages) allow for more attention to be paid
to the design, purpose, and real work of each thread rather than the mechanics of sending and receiving data.

This is clearly seen in \dcamp Service implementations where, for example, a single thread can be used to process both
remote nodes' performance metrics and the local node's performance metrics without any special coding: one socket with
multiple endpoints connected and the built-in fair-queuing taking care of routing.

\subsection{Quick Simulation}

The nature of ZMQ bind/connect endpoints just being a transport and a network address means simulation of large
distributed systems can take place on a single machine using unique port numbers. Additionally, the interoperability of
\texttt{inproc} (within the same process) and TCP transports allows for even larger simulations, not bound by port
availability on the host.

This is demonstrated several times in examples within \textit{The Guide}, and it indeed held true during \dcamp
development and testing.
