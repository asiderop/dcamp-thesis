%%%%%%%%%%%%%%%%%%%%%%%%%%%%%%%%%%%%%%%%%
% Beamer Presentation
% LaTeX Template
% Version 1.0 (10/11/12)
%
% This template has been downloaded from:
% http://www.LaTeXTemplates.com
%
% License:
% CC BY-NC-SA 3.0 (http://creativecommons.org/licenses/by-nc-sa/3.0/)
%
%%%%%%%%%%%%%%%%%%%%%%%%%%%%%%%%%%%%%%%%%

%----------------------------------------------------------------------------------------
%	PACKAGES AND THEMES
%----------------------------------------------------------------------------------------

\mode<presentation> {

% The Beamer class comes with a number of default slide themes
% which change the colors and layouts of slides. Below this is a list
% of all the themes, uncomment each in turn to see what they look like.

%\usetheme{default}
\usetheme{AnnArbor}
%\usetheme{Antibes}
%\usetheme{Bergen}
%\usetheme{Berkeley}
%\usetheme{Berlin}
%\usetheme{Boadilla}
%\usetheme{CambridgeUS}
%\usetheme{Copenhagen}
%\usetheme{Darmstadt}
%\usetheme{Dresden}
%\usetheme{Frankfurt}
%\usetheme{Goettingen}
%\usetheme{Hannover}
%\usetheme{Ilmenau}
%\usetheme{JuanLesPins}
%\usetheme{Luebeck}
%\usetheme{Madrid}
%\usetheme{Malmoe}
%\usetheme{Marburg}
%\usetheme{Montpellier}
%\usetheme{PaloAlto}
%\usetheme{Pittsburgh}
%\usetheme{Rochester}
%\usetheme{Singapore}
%\usetheme{Szeged}
%\usetheme{Warsaw}

% As well as themes, the Beamer class has a number of color themes
% for any slide theme. Uncomment each of these in turn to see how it
% changes the colors of your current slide theme.

%\usecolortheme{albatross}
%\usecolortheme{beaver}
%\usecolortheme{beetle}
\usecolortheme{crane}
%\usecolortheme{dolphin}
%\usecolortheme{dove}
%\usecolortheme{fly}
%\usecolortheme{lily}
%\usecolortheme{orchid}
%\usecolortheme{rose}
%\usecolortheme{seagull}
%\usecolortheme{seahorse}
%\usecolortheme{whale}
%\usecolortheme{wolverine}

%\setbeamertemplate{footline} % To remove the footer line in all slides uncomment this line
%\setbeamertemplate{footline}[page number] % To replace the footer line in all slides with a simple slide count uncomment this line

%\setbeamertemplate{navigation symbols}{} % To remove the navigation symbols from the bottom of all slides uncomment this line
}

\usepackage{graphicx} % Allows including images
\usepackage{booktabs} % Allows the use of \toprule, \midrule and \bottomrule in tables

%----------------------------------------------------------------------------------------
%	TITLE PAGE
%----------------------------------------------------------------------------------------

\title[dCAMP]{Distributed Common API for\\Measuring Performance} % The short title appears at the bottom of every slide, the full title is only on the title page

\author{Alexander P. Sideropoulos} % Your name
\institute[CalPoly] % Your institution as it will appear on the bottom of every slide, may be shorthand to save space
{
California Polytechnic State University, San Luis Obispo \\ % Your institution for the title page
\medskip
\textit{alexander@thequery.net} % Your email address
}
\date{\today} % Date, can be changed to a custom date

\begin{document}

\begin{frame}
\titlepage % Print the title page as the first slide
\end{frame}

\begin{frame}
\frametitle{Outline} % Table of contents slide, comment this block out to remove it
\tableofcontents[pausesections] % Throughout your presentation, if you choose to use \section{} and \subsection{} commands, these will automatically be printed on this slide as an overview of your presentation
\end{frame}

%----------------------------------------------------------------------------------------
%	PRESENTATION SLIDES
%----------------------------------------------------------------------------------------

%------------------------------------------------
\section{Introduction}
%------------------------------------------------

%------------------------------------------------
\subsection{Problem}
\begin{frame}
\frametitle{problem}
software increasingly distributed, need to test and evaluate performance of
distribute systems (page 1)
\end{frame}

%------------------------------------------------
\subsection{Solution}
\begin{frame}
\frametitle{solution}
dpf, define these (page 1-2)
\end{frame}

%------------------------------------------------
\subsection{Criterion}
\begin{frame}
\frametitle{criterion}
how to compare and choose
-   overview and sources (page 2)
-   list each one (page 2-4)
\end{frame}

%------------------------------------------------
\subsection{dCAMP}
\begin{frame}
\frametitle{dcamp}
introduced as a better dpf (page 5)
\end{frame}

%------------------------------------------------
\subsection{Terminology}
\begin{frame}
\frametitle{terms}
(page 5-9)
-   perf metric
-   metric sampling, reporting, aggregation, calculation
-   service + role
-   zeromq
-   zeromq address + endpoint
\end{frame}

%------------------------------------------------
\section{dCAMP Design}
%------------------------------------------------

%------------------------------------------------
\subsection{Architecture}
\frame{pipe and filter, hierarchically distributed}

%------------------------------------------------
\subsection{Roles and Services}
\frame{all of them with table}

%------------------------------------------------
\subsection{Metrics}
\frame{overview, dcamp metric groups, pitfall}

%------------------------------------------------
\subsection{Fault Tolerance}
\frame{heartbeating}
\frame{reminder}
\frame{promotion}
\frame{election}

%------------------------------------------------
\section{dCAMP Implementation}
%------------------------------------------------

%------------------------------------------------
\subsection{Operation}
\frame{high-level}
\frame{threading}

%------------------------------------------------
\subsection{Protocols}
\frame{topology}
\frame{configuration}
\frame{data}
\frame{recovery}

%------------------------------------------------
\section{dCAMP Analysis}
%------------------------------------------------

%------------------------------------------------
\subsection{Transparency}
\frame{workload}
\frame{configuration}
\frame{results}

%------------------------------------------------
\subsection{Scalability}
\frame{workload}
\frame{configuration}
\frame{results}

%------------------------------------------------
\section{Evaluation and Conclusion}
%------------------------------------------------

%------------------------------------------------
\subsection{dcamp}
\frame{dcamp}

%------------------------------------------------
\subsection{related work}
\frame{related work}

%------------------------------------------------
\subsection{contributions}
\frame{contributions}

%------------------------------------------------
\subsection{future work}
\frame{features}
\frame{fault tolerance}
\frame{performance and scalability}
\frame{metric extensions}

\subsection{References}
\begin{frame}
\frametitle{References}
\footnotesize{
\begin{thebibliography}{99} % Beamer does not support BibTeX so references must be inserted manually as below
\bibitem[Smith, 2012]{p1} John Smith (2012)
\newblock Title of the publication
\newblock \emph{Journal Name} 12(3), 45 -- 678.
\end{thebibliography}
}
\end{frame}

%----------------------------------------------------------------------------------------
%----------------------------------------------------------------------------------------

\appendix
\section{\appendixname}
\frame{\tableofcontents}

%------------------------------------------------
\subsection{ZeroMQ: Sockets and Patterns}
\frame{details}

%------------------------------------------------
\subsection{ZeroMQ: Messaging Patterns}
\frame{Publish/Subscribe}
\frame{Request/Reply}
\frame{Pipeline}
\frame{Exclusive Pair}

%------------------------------------------------
\subsection{dCAMP Messages}
\frame{Topology}
\frame{Configuration}
\frame{Data}

%------------------------------------------------
\subsection{dCAMP Protocols}
\frame{Topology}
\frame{Configuration}
\frame{Branch Recovery}
\frame{Root Recovery}

%------------------------------------------------
\subsection{dCAMP Configuration}
\frame{Node Specification}
\frame{Sample Specification}

\end{document} 
